\documentclass[UKenglish]{uiophdthesis}
\usepackage[utf8]{inputenc}
\usepackage[T1]{fontenc, url}  \urlstyle{sf}
\usepackage{babel, csquotes, fancyvrb, graphicx, textcomp, uiomasterfp, varioref}
%% \usepackage[backend=biber,style=numeric-comp]{biblatex}  %% No bibliography
\usepackage[hidelinks, hypertexnames=false]{hyperref}

\newcommand{\bsl}{\textbackslash}
\newcommand{\tettere}{\addtolength{\itemsep}{-1ex}}
\newcommand{\p}[1]{\textsf{#1}}
\newcommand{\pb}[1]{\textbf{\p{#1}}}
\newcommand{\pcmd}[1]{\p{\bsl #1}}
\newcommand{\ppar}[1]{\p{\{#1\}}}

\title{Writing your PhD thesis}
\subtitle{A guide to the \LaTeX{} document class \pb{uiophdthesis}}
\author{Dag Langmyhr\\ (\url{dag@ifi.uio.no})}

\begin{document}
\uiomasterfp[kind=Documentation, compact, nosp, colour=grey,
  date=\today]

\maketitle

\begin{abstract}
\LaTeX{} is an excellent tool for writing your PhD thesis, especially in
combination with the bibliography tool Bib\LaTeX.

The University of Oslo has defined guidelines\footnote{\label{guidelines}See
  \url{https://www.uio.no/english/services/print/doctoral-thesis/index.html}.}
for the PhD theses, and 
the University of Oslo Library and the Department of Informatics have
developed the document class \pb{uiophdthesis} in accordance with these
guidelines. This documentation has been typeset using that class.
\end{abstract}

\tableofcontents

\part*{Introduction}

\chapter{About the document class}
The \p{uiophdthesis} class has the following features:
\begin{itemize}
\item The paper size is A4, which is what the printers require (see
  the guidelines in footnote on page~\pageref{guidelines}).
  The thesis will be photographically reduced to
  81\,\% when printed.

\item The font size is 12\,pt which will look all right in the reduced
  print.

\item You should not attempt to create any kind of front page. This
  will always be supplied by the printers.\footnote{This documentation
    does have a front page because it provides technical information;
    it is not a thesis.}
\end{itemize}

\section{Fonts\label{font}}
The default document font for \p{uiophdthesis} is the standard \LaTeX{} font
{\fontfamily{lmr}\fontshape{it}\selectfont Computer Modern} in
combination with \textsf{\textit{Helvetica}}.
You may select a different font by using one of these
options to \pcmd{documentclass}:
\begin{description}
\item[\p{font=garamond}] selects
  {\fontfamily{mdugm}\fontshape{it}\selectfont Garamond} with
  \textsf{\textit{Helvetica}}
  
\item[\p{font=noto}] selects
  {\fontfamily{NotoSerif-LF}\fontshape{it}\selectfont Noto serif} with
  {\fontfamily{NotoSans-TLF}\fontshape{sl}\selectfont Noto sans}

\item[\p{font=times}] selects
  {\fontfamily{ptm}\fontshape{it}\selectfont Times Roman} with
  \textsf{\textit{Helvetica}}
\end{description}

\section{Document structure}
We recommend that you use the \textbf{\pcmd{part*}\ppar{\dots}}
command to structure your thesis. (If you want numbered parts, you may
omit the \pb{*}.)

\subsection{Included research papers}
Your thesis is likely to contain one or more research paper published
in various publications. When you want to include such a research
paper, you should use the command \textbf{\pcmd{uiopaper}} rather than
a \pcmd{part} or \pcmd{part*}. This will affect the numbering.

Many research papers were originally written in the \p{article} style
which has no \pcmd{chapter} command. To make it easier to include
these in the thesis, the \pcmd{uiopaper} command may be given as
\begin{quote}
  \p{\pcmd{uiopaper}[nochapter]\ppar{\emph{paper title}}}
\end{quote}
Then, the paper will be properly typeset with no \pcmd{chapter}
commands but \pcmd{section} as the highest structuring command.

If you want to include the research paper exactly as it was typeset,
you can import it from a \textsc{pdf} file using the command
\textbf{\pcmd{uioincludepdf}} giving the \textsc{pdf} file as
parameter. 

\chapter{Installation}
If you are processing your \LaTeX{} document on a stationary Linux
computer at the University of Oslo, you need not worry about
installing the \p{uiophdthesis} document class; it is already there.

\section{On your personal computer\label{privat-pc}}
To use this document class on your own computer (which may run Linux, MacOS
or Windows) you must do the following:
\begin{enumerate}
\item Fetch
  \url{https://www.mn.uio.no/ifi/tjenester/it/hjelp/latex/uiotheses.zip}. (Click
  on the \textsc{url} to download the file.)
  
\item Unzip the files. You may place all the files in the same folder
  as your \LaTeX{} source files.\footnote{If you know where \LaTeX{}
    packages are kept on your computer, you 
    can save them there to make them generally available. Remember to
    refresh your file name database afterwards.}
\end{enumerate}
And that should be all.

\section{Using Overleaf}
If you are using Overleaf (see \url{https://www.overleaf.com}) to
write your thesis, you may do the following to use the \p{uiophdthesis}
document class:
\begin{enumerate}
\item Fetch 
  \url{https://www.mn.uio.no/ifi/tjenester/it/hjelp/latex/uiotheses.zip}. (Click
  on the \textsc{url} to download the file.)

\item Unpack the \textsc{zip} file.\footnote{Overleaf allows import of
  \textsc{zip} files, but \emph{only} if it is the first thing you do
  after creating a new project.}

\item In your Overleaf project, select the upload icon
  (``\includegraphics[height=2ex]{upload}''). Then, select all the
  unzipped files and upload them.
\end{enumerate}
Once this has been done, you may use the document class.

\chapter{Using the document class}
To use this document class, just start your \LaTeX{} file with
\begin{quote}
  \pcmd{documentclass[\emph{options}]}\ppar{uiophdthesis}
\end{quote}
The option \pb{font=\emph{xxx}} is used to select your font; see
Section~\vref{font}. All other options are passed to packages you use.

\section{An example}
The \p{uiophdthesis} package comes with a base file named
\pb{uiophdthesis-base.tex} containing the basic layout of your thesis;
see Figure~\vref{fig:base}. The idea is that you make a copy of that
file, modify the specified texts, and then write your thesis.

\begin{figure}[htp]
  \VerbatimInput[fontsize=\footnotesize,frame=single,obeytabs,
    label={\pb{uiophdthesis-base.tex}},numbers=left]{uiophdthesis-base.tex}
  \caption{The file \p{uiophdthesis-base.tex}\label{fig:base}}
\end{figure}

\begin{description}
\item[Line 1:] The document class should be \pb{uiophdthesis}. You must
  also specify the language of your thesis.
\item[Line 2:] UTF-8 is the most common character encoding in use
  today, so, unless you specify otherwise in your text editor, you are
  likely to get this encoding.
\item[Line 3:] The \p{url} package provides the \pcmd{url}
  command which is very useful for typesetting long internet
  addresses. These should be set in a \textsf{sans serif} typeface
  (rather than \texttt{teletype}). For an example, see
  Section~\vref{privat-pc}.
\item[Lines 4--6:] These packages should always be included:
  \begin{description}
  \item[\p{babel}] handles language adaption.
  \item[\p{csquotes}] supports quote marks in various language. This
    package is required by \p{biblatex}; see below.
  \item[\p{graphicx}] provides support for including illustrations.
  \item[\p{textcomp}] adds many useful symbols.
  \item[\p{varioref}] gives improved features for crossrefererencing.
  \item[\p{biblatex}] loads Bib\LaTeX{} which handles
    bibliographies.\footnote{\emph{Local guide to Bib\LaTeX} at
      \url{https://www.mn.uio.no/ifi/tjenester/it/hjelp/latex/biblatex-guide.pdf}
      is a simple introduction to creating your bibliography.}
    The package options given here are recommended; they use the
    numeric citation style favoured in natural science.
  \item[\p{hyperref}] provides hyperlinks both internally and externally.
  \end{description}

\item[Lines 8--9:] You must always state a thesis title. You may also
  provide a subtitle, but that is not mandatory.
\item[Line 10:] Don't forget you own name!

\item [Line 12:] \pcmd{addbibresouce} specifies the name/s of your Bib\LaTeX{}
  bibliography file/s.

\item[Line 15:] specifies the start of the thesis front matter, i.e.,
  abstract, table of contents~etc.
\item[Line 16:] prints an inner title page. (This will appear after
  the front page so it is recommended.) Normally, just the title, subtitle and
  author's name is printed, but you may use these options to
  \pcmd{maketitle} for additional information:
  \begin{description}
  \item[\p{dept=\ppar{\emph{department name}}}] gives the name of the
    department. 
  \item[\p{fac=\ppar{\emph{faculty name}}}] gives the faculty name. 
  \item[\p{supervisor=\ppar{\emph{supervisor's name}}}] names the
    supervisor. 
  \item[\p{supervisors=\ppar{\emph{supervisor's name} \pcmd{and} \emph{name}
          \pcmd{and} \emph{name}}}] gives the names of the
    supervisors, if there are more than one. 
  \end{description}
  You may also use the standard \pcmd{date} command to specify the date.
\item[Lines 23--25:] contains your abstract.
\item[Lines 26--29:] contains your abstract in a different language,
  for example in Norwegian. \textbf{A Norwegian abstract is mandatory
    at the University of Oslo.} 
\item[Lines 31--33:] produces your tables of content, figures and
  tables, accordingly.
\item[Lines 35--37:] is your preface.

\item[Line 39:] shows the start of the main part of your thesis.
\item[Line 40--42:] shows your thesis structure: \pcmd{part},
  \pcmd{chapter}, \pcmd{section}, \pcmd{subsection}~etc.
  Use the *-ed form for unnumbered headings.
\item[Line 44--46:] contains a research paper typeset as part of the thesis.
\item[Line 48--49:] contains a different research paper avaible as a
  \textsc{pdf} file. 
\item[Line 51--53:] is an additional thesis text not part of a paper.
\item[Line 55:] starts the back part containing appendices,
  bibliography and such.
\item[Line 56:] prints the bibliography created by Bib\LaTeX.
\end{description}

\section{Another example}
The file \url{uiophdthesis-guide.tex} shows the \LaTeX{} source
code for this documentation.

\end{document}
